\section{Physic's Model Study}
The haptic device we tried to develop for this thesis is at its heart a multi-physical system, where electrical, magnetic, and mechanical components interact with each other.
Voice coil actuators are characterized by three main components: a coil, a magnet, and a membrane. The coil generates a magnetic field when a current flows through it, which interacts with the magnetic field generated by the magnet, creating a force that moves the membrane. The membrane is then used to transmit this force to the user's finger, which can feel the force as a vibration.
In this section, we will try to create a complete model to explain its behavior.
\begin{figure}[H]
    \centering
    \resizebox{1\linewidth}{!}{
        \begin{tikzpicture}
    \begin{scope}[every node/.style={bgelement}]
    \node (Se) at (0,0) {Se: Power Supply};
    \node[right=1 of Se] (i) {1: elec};
    \node[above=1 of i] (Iel) {I: L\textsubscript{coil}};
    \node[below=1 of i] (Rel) {R: R\textsubscript{coil}};
    \node[right=1 of i] (CI) {CI: Coil+Magnet};
    \node[right=1 of CI] (w1) {1: mech};
    \node[above=1 of w1] (Cm) {C: C\textsubscript{membrane}};
    \node[below=1 of w1] (Im) {I: M\textsubscript{magnet} + M\textsubscript{finger}};
    \node[right=1 of w1] (v) {0};
    \node[above=1 of v] (w2) {1};
    \node[above=1 of w2] (Rdf) {R: Rd\textsubscript{finger}};
    \node[right=1 of w2] (Cf) {C: $\frac{1}{Ks_{finger}}$};
    \node[right=1 of v] (Sfa) {S\textsubscript{finger}};
    \end{scope}
    \draw[bonds]
    (Se) edge [e_out] (i)
    (i) edge [e_in] (Iel)
    edge [e_in] (Rel)
    edge [e_in] (CI)
    (CI) edge [e_in] (w1)
    (w1) edge [e_out] (Cm)
    edge [e_in] (Im)
    (w1) edge [e_in] (v)
    (v) edge [e_in] (w2)
    (w2) edge [e_in] (Rdf)
    (w2) edge [e_in] (Cf)
    (Sfa) edge [e_out] (v);
\end{tikzpicture}
    
    } % TODO: Da fare bene
    \caption{Bond graph of the coil-magnet-membrane system.}
    \label{fig: Total_bond-graph}
\end{figure}
The best way to model this system is by using a bond graph.

\subsection{Electrical and Power Aspects of Coils}
The electrical part consists of the voltage power supply and the coil which is modeled as a resistor and an inductor in series. The coils we're using are planar concentric coils built using flexible PCB technology, this allows us freedom of design in creating a flexible actuator.
Due to their nature, these coils are very thin, which results in high resistance and high heat production. The power flowing through the coil must be kept low to avoid damaging it. This in turn means that the magnetic field generated by the coil will be weak, which can be a problem when trying to create strong haptic feedback.

\subsection{Electro-mechanical Transducer}
The electrical energy is converted into mechanical motion through the magnetic repulsion force between coil and magnet, so the system acts as a transducer.
\begin{figure}[H]
    \centering
    \includegraphics[width=0.4\columnwidth]{Figures/coil_magnet.png} 
    \caption[Coil-Magnet position]{Coil and magnet position in space.}
\end{figure}
The force is calculated through the magnetic levitation formula between the magnetic moment of the magnet and the magnetic field generated by planar windings:
\begin{equation*}
    F = \nabla (\overrightarrow{m_M} \cdot \overrightarrow{B_C}) = \frac{d}{dz} \left( \frac{B_M(z)}{\mu} \cdot \frac{\mu N I R_C^2}{2(R_C^2+z^2)^\frac{3}{2}} \right)
\end{equation*}

\subsection{Mechanical Membrane}
To limit the motion of the magnet only to the z-axis the magnet is inserted in a flexible membrane, this membrane is also used to transmit the force generated by the magnet to the user's finger.
For the last prototype, we designed a celtic-cross-shaped silicone membrane where the magnet is inserted in the center of the cross.
\begin{figure}[H]
    \centering
    \includegraphics[width=0.9\linewidth]{Figures/membr_mech_model.jpg} 
    \caption[Membrane structure]{Membrane structure and mechanical two fixed beams model of the last prototype.}
\end{figure}
The membrane can be modeled as a spring-damper system, where the spring represents the stiffness of the membrane, and the damping is negligible.
Finally, the finger grasping the membrane can also be modeled as a \textbf{mass-spring-damper} system where the inertia component is represented by the finger's mass, while the spring and damper represent respectively the stiffness and damping effect of the finger's skin.
