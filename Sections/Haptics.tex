\section{Haptics Overview}
Haptic Feedback is the research field that deals with the need to be able to \textbf{digitalize the human sense} of touch and \textbf{reproduce it}.
As this sense is \textbf{very complex} we still don’t understand it fully and we are \textbf{still not able to emulate it completely}. 
The human tactile sensing system can \textbf{measure specific properties of materials}, such as temperature, texture, force, friction, hardness and viscoelasticity, \textbf{through physical contact} between the human skin and the object.
Even the \textbf{changing state of the interaction}, such as gravitational and inertial effects, can be perceived through the sense of touch.
The sense of touch is based on the somatosensory system, which is composed of four main types of receptors, each one is specialized in detecting a specific type of stimulus: \textbf{mechanoreceptors, thermoreceptors, nociceptors and proprioceptors}.
Mechanoreceptors are the most important receptors for haptic feedback, as they are responsible for detecting \textbf{vibration and pressure}.
Their \textbf{sensitivity is limited} in both the frequency and amplitude of vibrations they can detect. The force amplitude limit also depends on the \textbf{constant pressure} applied to the skin.