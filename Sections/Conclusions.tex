\section{Conclusions}
\subsection{Low force Output}
In the previous chapter, we measured the \textbf{maximum force} our system could generate in the \textbf{best conditions}.
Even by \textbf{overloading} the coil with short bursts of 30V it couldn't generate \textbf{enough force} to reach the \textbf{haptic sensitivity of human fingers}. Test subjects were anyway able to feel some vibrations, we can theorize that the sensitivity \textbf{threshold was lowered by the constant pressure} force applied on the skin by the membrane.
Even still the vibrations were \textbf{very weak} and \textbf{barely perceptible}.
\subsection{A technology not suitable for haptic feedback devices}
Driving flexible coils with \textbf{AC signals} is very \textbf{difficult} and \textbf{inefficient}.
The complex circuitry required to drive these coils is not only \textbf{expensive} but also \textbf{difficult to design and tune}.
The high power consumption is also a major drawback.
Combining all these factors with the low magnetic field output, we can conclude that this technology is \textbf{not ready for mainstream application in haptic feedback devices}.