\section{Conclusions}
\subsection{Low force Output}
In the previous chapter, we measured the maximum force our system could generate in the best conditions.
Even overloading the coil with short bursts of 30V it couldn't generate enough force to reach the haptic sensitivity of human fingers. Test subjects were anyway able to feel some vibrations, we can theorize that the sensitivity threshold was lowered by the constant pressure force applied on the skin by the membrane.
Even still the vibrations were \textbf{very weak} and \textbf{barely perceptible}.
\subsection{A technology not suitable for haptic feedback devices}
As we discussed multiple times in this thesis, driving these low-resistance flexible coils with \textbf{AC signals} is very \textbf{difficult} and \textbf{inefficient}.
The complex circuitry required to drive these coils is not only \textbf{expensive} but also \textbf{difficult to design and tune}.
The high power consumption is also a major drawback.
Combining all these factors with the low magnetic field output, we can conclude that this technology is \textbf{not ready for mainstream application in haptic feedback devices}.